\section{Четверта практика}\marginpar{\framebox{09.04.2014}}
\begin{tsk}
Середні норми витрат для фабрик та максимальні об’єми виробництва
\begin{center}
\begin{tabular}{|c|cccc|c|}
\hline
Види ресурсу & \multicolumn{4}{|c|}{Норми витрат} & Об’єм\\
\hline
Сировина & 4 & 5 & 2 & 5  & 80 \\
Робоча сила & 20 & 12 & 20 &40 & 400 \\
Обладнання & 10 & 15 & 10 & 16 & 150 \\
\hline
\end{tabular}
\end{center}
Середній прибуток з кожної фабрики:
\begin{center}
\begin{tabular}{|c|c|c|c|c|}
\hline
i & 1 & 2 & 3 & 4 \\
\hline
$c_i$ & 30 & 25 & 56 & 8 \\
\hline
\end{tabular}
\end{center}
Нечіткі величини, середні значення яких вказані на таблиці, мають такі розподіли.
\begin{eqnarray}
&\mu_\gam(\aij) = e^{\cfrac{\cb{\aij - \bar{\aij}}^2}{2}}\\
&\gamma\cb{c_j} = \cfrac 1{1 - \cb{c_j-\bar{c_j}}^2}
\end{eqnarray}
Запишемо математичну модель недетермінованої задачі
\begin{eqnarray}
&\max\sum_{j} c_jx_j\\
&\suml_{j} \aij\xj \leq v_i\\
&x_j\geq 0\\
&\mu\cb{\aij} \geq 0.8\\
&\mu\cb{c_j} \geq 0.8\\
\end{eqnarray}
Розв’яжемо дві останні нерівності для оцінки можливих варіантів:
\begin{eqnarray}
&\cfrac12 \geq \mdl{c_j - \bar{c_j}} \\
&-\cfrac12 + \bar {c_j} \leq c_j \leq \cfrac12 + \bar{c_j}\\
&\mdl{\aij - \bar{\aij}} \leq \sqrt{-2\ln 0.8} \\
&-\sqrt{2\ln\cfrac54} + \bar{\aij} \leq \aij \leq \sqrt{2\ln\cfrac54} + \bar{\aij}
\end{eqnarray}
Запишемо задачі песиміста та задачі оптиміста, тобто задачі, які передбачають найгірший та найкращій розвиток:\\
Задача песиміста:
\begin{eqnarray}
&\max \suml_{j} \cb{\bar{c_j} -\cfrac12} x_j\\
&\suml_j \cb{\bar{\aij} +\sqrt{2\ln\cfrac54}} x_j \leq v_i
\end{eqnarray}
Задача оптиміста:
\begin{eqnarray}
&\max \suml_{j} \cb{\bar{c_j} +\cfrac12} x_j\\
&\suml_j \cb{\bar{\aij} -\sqrt{2\ln\cfrac54}} x_j \leq v_i
\end{eqnarray}
\end{tsk}
 