\section{Двухетапні задачі стохастичного математичного програмування}\marginpar{\framebox{19.02.2014}}
В задачах ПР при наявності випадкових факторів, які мають детерміновані обмеження і випадкові, процес розв’язку розбивається на два етапи:
\begin{enumerate}
\item Визначається за детермінованими обмеженнями і далі спостерігаємо реалізацію випадкових обмежень задачі;
\item Примаємо наступні розв’язки як план компенсації та виявлення нев’язок.
\end{enumerate}
\begin{eqnarray}
\min \mEt{ \suml_{j=1}^n \cj\cb{\omg} \xj} = \mt{\bar c} x\label{tr:2:3}\\
A^{(1)} x = B^{(1)} \label{tr:2:1}\\
A(\omg) X = B(\omg) \label{tr:2:2}\\
x\geq 0\label{tr:2:4}
\end{eqnarray}
Де \eqref{tr:2:1} - детерміновані обмеження, а \eqref{tr:2:2} - випадкові обмеження. 
$A=\mdl{\mdl{\aij(\omg)}}_{\ifom}^{\jfon}$ - випадкові.\\
$B(\omg) = \bb{ \bi(\omg)}_{\ifom}$\\
$A^{(1)} = \mdl{\mdl{\aij}}_{\ifom}^{\jfon}$\\
$\mEt{\cj} = \bar{\cj}$\\
Маємо задачу \eqref{tr:2:3}-\eqref{tr:2:4}. Будемо розв’язувати в два етапи.\\
\subsubsection{Перший етап}
\begin{eqnarray}
&\min\mEt{\suml_{j=1}^n \cj(\omg) \xj} \\
&A^{(1)} X = B^{(1)} \\
&x\geq 0 
\end{eqnarray}
Розв’язавши цю задачу отримаємо план $X$, який буде детермінованим. Нев’язка $\eps = B(\omg) - A(\omg) x$\\
\subsubsection{Другий етап}
План компенсації нев’язко $y=[y_i]_{\ifom}$\\
$D(\omg)$ - матриця компенціючих способів виробництва.\\
\begin{equation}
Dy = B-Ax
\end{equation}
Нехай вектор 