\section{Складання математичних моделей}\marginpar{\framebox{26.02.2014}}
\subsection{Задача №2}
Таблиця рівномірного розподілу затрати ресурсів:\\
\begin{tabular}{|l|l|l|l|l|l|}
\hline
Дошки & Стіл & Бюро & Стул & Шкаф & Кількість\\
\hline
Перший ресурс & \bb{4,6} & \bb{8,12} & \bb{1,2} & \bb{9,15} & 1000\\
\hline
Другий ресурс &  \bb{1,3} & \bb{4,6} & \bb{2,4} & \bb{10,16} & 500 \\
\hline
Чоловіко-ресурс & \bb{2,5} & \bb{5,8} & \bb{1,3} & \bb{8,12} & 800\\
\hline
Ціна & 120 & 150 & 50 & 300& \\
\hline
\end{tabular}\\
Початок інтервалу позначимо $\gamij$, а кінець $\delij$.\\
Потрібно знайти розв’язок, який задовільняє нас з ймовірністю більше, ніж $p=0.9$\\
Запишемо математичну модель:

\begin{eqnarray}
&\max\set{120x_1+150x_2+50x_3+300x_4}\\
&\mP\set{\suml_{j=1}^3 \aij \xj \leq \bi}  \geq \alij
\end{eqnarray}
Перетворимо у детермінований еквівалент.

\begin{equation}
5x_1+10x_2+\cfrac32 x_3+12x_4 + \Phi^{-1}\cb{0.9}\cb{\cfrac13x_1^2+\cfrac43x_2^2+\cfrac1{12}x_3^2+3x_4^2}\leq 1000
\end{equation}
Інші аналогічно.\\
Вводимо додаткову умову з задачі: $\cfrac{x_1}{x_3}=\cfrac16\Rightarrow x_3 = 6x_1$\\
Отримаємо змінену математичну модель:
\begin{eqnarray}
&\max\set{120x_1+150x_2+300x_3+300x_4}\\
&5x_1+10x_2+9x_1 + 12x_4 + \Phi^{-1}\cb{0.9}\cb{\cfrac13x_1^2+\cfrac43x_2^2+3x_1^2+3x_4^2}\leq 1000
\end{eqnarray}
\subsection{Завдання №3}
Матриця витрату ресурсів: \\
\begin{tabular}{|c|c|c|c|c|c|c|c|c|}
\hline 
 & \multicolumn{6}{c|}{Норми трат} & \multicolumn{2}{c|}{Об’єм} \\ 
\cline{2-7}
Продукти& \multicolumn{2}{c|}{A} & \multicolumn{2}{c|}{B} & \multicolumn{2}{c|}{C} &\multicolumn{2}{c|}{$\phantom{1}$}\\ 
\cline{2-9}
 & 1 & 2 & 1 & 2 & 1 & 2 & 1 &2\\ 
\hline
1 & \bb{1,3} & \bb{2,6} & \bb{0.5,1.5} & \bb{2,3} &\bb{ 2,3} & \bb{2,4} & 250 & 150\\ 
\hline 
2 & \bb{1,2} & \bb{3,7} & \bb{1,3} & \bb{1.5,3} & \bb{2,2.5} & \bb{1,4} & 100 & 200\\ 
\hline 
3 & \bb{1,3}& \bb{2,4} & \bb{1,1.5} & \bb{1,2} & \bb{1,3} & \bb{3,5} & 240 & 300\\
\hline
& \multicolumn{2}{c|}{$x_1$} & \multicolumn{2}{c|}{$x_2$} & \multicolumn{2}{c|}{$x_3$} && \\
\hline
&\multicolumn{2}{c|}{300} & \multicolumn{2}{c|}{170} & \multicolumn{2}{c|}{250}&& \\
\hline
\end{tabular} \\
Таблиця вартості: \\
\begin{tabular}{|c|c|c|c|}
\hline
$\cij$& A & B & C \\
\hline
1 & 2 & 8 & 5 \\
\hline
2 & 3 & 6 & 6\\
\hline
3 & 3 & 9 & 5\\
\hline
\end{tabular}\\
Відповідно, обмежуюча ймовірність $p=0.9$. Потрібно мінімізувати витрати. $\xij$ - скільки виробляємо $j$-того товару на $i$-тому виробництві.\\
\begin{eqnarray}
&\min\set{\suml_{i,j}\cij\xij} \\
&\mP\set{\suml_{j=1}^3\aij^{(k)}\xij \leq \bi^{(k)}}\geq 0.9,\ifo3,\kfo2
\end{eqnarray}
Детермінізуємо. З критерієм все очевидно, бо він не змінюється, запишемо обмеження.\\
Для $k=1$:\\
\begin{eqnarray}
2x_{11} + x_{12} + 2.5x_{13} +\Phi^{-1}\cb{0.9}\cb{\cfrac13 x_{11}^2 +\cfrac1{12} x_{12}^2 + \cfrac1{12}x_{13}^2}^{\frac12} \leq 250
\end{eqnarray}\\
$k=2,i=1$:\\
\begin{equation}
4x_{11} + 2.5x_{12} + 3x_{13} + \Phi^{-1}\cb{0.9}\cb{\cfrac43x^2_{11}+\cfrac1{12}x_{12}^2+\cfrac23 x_{13}^2}^{\frac12} \leq 150
\end{equation}
Також додаються обмеження на кількість продукцій:\\
\begin{eqnarray}
&x_{11}+x_{21}+x_{31} \geq 300\\
&x_{12}+x_{22}+x_{32} \geq 170\\
&x_{13}+x_{23}+x_{33} \geq 250
\end{eqnarray}
