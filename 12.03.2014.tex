\section{Нечітке відношення нестрогої переваги} \marginpar{\framebox{12.03.2014}}
\textbf{Нечітким відношенням нестрогої переваги} $R$ називається будь-яке рефлексивне нечітке відношення. \\
Відношення $R$ породжує \textbf{нечітке відношення строгої переваги} $R_s= R\setminus R^{-1}$.
\begin{equation}
\mu_{R_s}(x,y) = \max\set{\mu_R(x,y) - \mu_R(y,x),0}
\end{equation}
$\mu_{R_s}(x,y)$ - ступінь, з якої $x$ домінує $y$.\\
Також $R$ породжує \textbf{нечітке відношення еквівалентності} $R_e = R\cup R^{-1}$\\
\begin{equation}
\mu_{R_e}(x,y) = \min\set{\mu_R(x,y),\mu_R(y,x)}
\end{equation}
\textbf{Нечітке відношення індеферентності} $R_i = \cb{X\times X}\setminus\cb{R\cup R^{-1}}\cup\cb{R\cap R^{-1}}$ 
\begin{equation}
\mu_{R_i} (x,y) = \max\set{1 - \max\set{\mu_R(x,y),\mu_R(y,x)},\min\set{\mu_R(x,y),\mu_R(y,x)}}
\end{equation}	
В тому випадку, якщо $R^{-1}\cup R = X\times X$, то відношення еквівалентності співпадає з відношенням індеферентності.\\
Задана множина альтернатив $X$, задано $R$ - нечітке відношення нестрогої переваги. Воно відразу породжує $R_s,R_e,R_i$.\\
$1-\mu_{R_s}(y,x)$ - степінь, з якої $x$ не домінується $y$.\\
Спробуємо знайти степінь, з якої $x$ не домінується будь-якою альтернативою $y$.\\
\begin{equation}
\mu_R^{nd}(x) = \max\limits_{y\in X} \set{1-\mu_{R_s}(y,x)}
\end{equation}
\textbf{Множиною недомінованих альтернатив} за нечітким відношенням нестрогої переваги $R$ називається нечітка множина $R^{nd}$ для якої:
\begin{equation}
\mu_R^{nd}(x) = \max\limits_{y\in X} \set{1-\mu_{R_s}(y,x)} = \min\limits_{y\in X} \set{\mu_R(y,x) - \mu_R(x,y)}
\end{equation}
Якщо $\mu_R^{nd}(x) = 1$, то $x$ це ЧНД (чітко недомінована) альтернатива.\\
\textbf{Множиною чітко недомінованих альтернатив }за нечітке відношення нестрогої переваги $R$ називається множина (чітка) $X_{cnd}$, для якого виконується:
\begin{equation}
X_{cnd} = \set{x:\mu_{R^{nd}}(x)=1}
\end{equation}
Нечітке відношення нестрогої переваги $R$ називається \textbf{сильно лінійним}, якщо:
\begin{equation}
\max\set{\mu_R(x,y),\mu_R(y,x)}=1,\forall x,y\in X
\end{equation}
Нечітке відношення нестрогої переваги $R$ називається \textbf{$\la$-лінійним} $\cb{\la\in\bb{0,1}}$, якщо:
\begin{equation}
\max\set{\mu_R(x,y),\mu_R(y,x)}\geq\la,\forall x,y\in X
\end{equation}
\begin{teor}
Якщо нечітке відношення нестрогої переваги $R$ є сильно лінійним, то ЧНД альтернативи $x_1$ та $x_2$ є еквівалентні зі степіню 1. Тобто $\mu_{R_e}(x_1,x_2) = 1$
\end{teor}
\section{Прийняття рішень за кількома критеріями}
Задана множина альтернатив $X$, на якій задано кілька відношень нестрогої переваги (нечітких або чітких), які ми будемо позначати $R_j,\jfom$. Також задані ваги критерії $w_j\geq 0,\suml_{j=1}^m w_j = 1$. Потрібно прийняти найкраще рішення.\\
Зробимо так:
\begin{enumerate}
\item $Q_1 = \bcapl_{j=1}^m R_j;\mu_{Q_1}(x,y) = \min\set{\mu_{R_j}(x,y)}$;
\item Породжується $Q_{1,s}$ та $\mu_{Q_{1,s}}(x,y) = \max\set{\mu_{Q_1}(x,y) -\mu_{Q_1}(y,x),0}$;
\item Знаходимо $Q_1^{nd} = 1-\max\limits_{y\in X}\set{\mu_{Q_{1,s}}(x,y)}$;
\item $x_0:\mu_{Q_1^{nd}}(x_0) = \max\set{\mu_{Q_1^{nd}}(x)}$;
\item Якщо альтернатив багато, то будуємо ще одну згортку, яка є взваженою сумою наших критеріїв
\begin{equation}
Q_2 = \suml_{j=1}^m w_jR_j
\end{equation}
\begin{equation}
\mu_{Q_2}(x,y) = \suml_{j=1}^m w_j\mu_{R_j}(x,y)
\end{equation}
\item Породжується $Q_{2,s}$ та $\mu_{Q_{2,s}}(x,y) = \max\set{\mu_{Q_2}(x,y) -\mu_{Q_2}(y,x),0}$;
\item Знаходимо $Q_2^{nd} = 1-\max\limits_{y\in X}\set{\mu_{Q_{2,s}}(x,y)}$;
\item Після цього шукає найкращу альтернативу за обома згортками.
\item $\mu_Q^{nd}(x) = \min\set{\mu_{Q_1}^{nd}(x),\mu_{Q_2}^{nd}(x)}$;
\item $x_0:\mu_{Q^{nd}}(x_0) = \max\set{\mu_{Q^{nd}}(x)}$;
\end{enumerate}
\subsection{Підхід Белмана-Заде}
Знаходження розв’язку при нечітких критеріях та нечітких обмеженнях.\\
Постановка задачі: є множина альтернатив $X$. Нехай на цій множині задана нечітка ціль $G$ з функцією приналежності $\mu_G(x)$ і нечітке обмеження $C$ з функцією приналежності $\mu_C(x)$.\\
Суть метода в тому, що не робиться ніякої різниці між цілю та обмеженнями, вони розглядаються як "полноправние" нечіткі множини. Тоді нечітка множина розв’язків визначається як $D = G \cap C:\mu_D(x) = \min\set{\mu_C(x),\mu_G(x)}$. \\
Тобто, шукаємо $x_0:\mu_D(x_0) = \max\set{\mu_D(x)}$\\
Це можна узагальнити на випадок, коли є кілька критеріїв та кілька обмежень. Нехай є $m$ цілій $G_1,\ldots,G_m$ та $n$ нечітких обмежень $C_1,\ldots,C_n$.\\
Тоді, множина розв’язків 
\begin{eqnarray}
D &=& G_1\cap \ldots\cap G_m \cap C_1\cap\ldots C_n\\
\mu_D(x) &=& \min\limits_{i,j} \set{\mu_{C_j}(x),\mu_{G_j}(x)}
\end{eqnarray}
Тоді найкраща альтернатива $x_0$:
\begin{equation}
\mu_D(x_0) = \max\mu_D(x)
\end{equation}
\subsection{Класифікація задач нечіткого математичного програмування}
\begin{description}
\item[Клас 1.] Чітка ціль $f(x)$ та нечітке обмеження $C,\mu_C(x)$. Тоді знаходимо $\max f(x) = f_{\max},\min f(x) = f_{\min}$\\
І далі переходимо від $f(x)$ до $\phi(x) = \cfrac{f(x) - f_{\min}}{f_{\max} - f_{\min}}\in\bb{0,1}$\\
Далі використовуємо підхід Белмана-Заде;
\item[Клас 2.] Чітка ціль та чіткі обмеження.
\begin{eqnarray}
&\max f(x)\\
&g_i(x)\leq 0,\ifom
\end{eqnarray} 
%деякий малюнок
Задаємося деякою границею $z$ та переходимо від цієї задачі до 
\begin{eqnarray}
&f(x)\to z\\
&g_i(x) \leq 0,\ifom
\end{eqnarray}
Введемо поріг $a$.Тоді ми можемо ввести таку функцію приналежності
\begin{equation}
\mu_f(x) = \system{1,f(x)=z\\0,f(x)<z-a\\>0,z>f(x)\geq z-a}
\end{equation}
%графік цієї штуки
Можемо записати цю функцію однозначно:
\begin{equation}
\mu_f(x) = \cfrac{f(x)-(z-a)}{a}
\end{equation}
Тепер розглянемо обмеження. Нехай ми можемо порушити ці обмеження з якимсь порогом $b_i$. Тобто
\begin{equation}
0\leq g_i(x) \leq b_i
\end{equation}
Введемо функцію приналежності
\begin{equation}
\mu_{g_i}(x) = \system{1,g_i(x)\leq 0 \\ 0, g_i(x)\geq b_i \\ >0,0<g_i(x)<b_i}
\end{equation}
%графік
Можемо записати цю функцію однозначно:
\begin{equation}
\mu_{g_i}(x) = \cfrac{b_i - g_i(x)}{b_i}
\end{equation}
Далі використовуємо підхід Белмана-Заде
\item[Клас 3.] Нечітка ціль $\phi(x)=y;\mu_\phi(x,y):X\times Y\to\bb{0,1}$. Та нечіткі обмеження $C;\mu_C(x)$;
\item[Клас 4.] Чітка ціль та нечіткі обмеження, де нечіткі саме параметри обмеження. 
\begin{eqnarray}
&\max f(x) = \suml_{j}\cj\xj\\
&g_i(x) = \suml_{j=1}^n \aij\xj\leq \bi,\ifom\\
&x_j\geq 0,\jfon
\end{eqnarray}
де $\aij$ - нечіткі величини;
\item[Клас 5.] Нечітка ціль та нечіткі обмеження, де нечіткість в параметрах
\begin{eqnarray}
&\max f(x,C) = \suml_{j}\cj\xj\\
&g_i(x) = \suml_{j=1}^n \aij\xj\leq \bi,\ifom\\
&x_j\geq 0,\jfon
\end{eqnarray}
$\cj,\aij$ - нечіткі величини. Саме такими задачами ми й будемо займатися.
\end{description}