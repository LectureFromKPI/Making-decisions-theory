\section{Багатокритерійні задачі прийняття рішень}\marginpar{\framebox{02.04.2014}}
\begin{eqnarray}
\max f_i(x),&i\in I_1\\
\min f_i(x),&i\in I_2\\
g_i(x) \leq b_i,&\ifom
\end{eqnarray}
Вектор $x^0$ називається \textbf{парето оптимальний розв’язком}, якщо $\not\exists x': $
\begin{eqnarray}
f_i(x') \geq f_i(x^0),&i\in I_1\\
f_i(x') \leq f_i(x^0),&i\in I_2
\end{eqnarray}
І хоча б одна нерівність виконується строго. Тобто $\not\exists x'$, який би домінував над $x^0$.\\
Альтернативи $x_1,x_2$ називаються непорівнюваними, якщо за деякими критеріями одна краща, а за іншими - інша.\\
\begin{tver}
Всі парето-оптимальні альтернативи або еквівалетні, або непорівняні.
\end{tver}
\begin{equation}\label{tr:8:1}
x_1 \geq x_2 \Leftrightarrow \system{f_i(x_1)\geq f_i(x_2),i\in I_1\\f_i(x_1)\leq f_i(x_2),i\in I_2}
\end{equation}
$x_1 > x_2$, якщо виконується \eqref{tr:8:1} і хоча б одна нерівність виконується строго.\\
Вектор критеріїв $f^{(1)}(x) = \bb{f_i^{(1)}(x)}$ еквівалентний $f^{(2)}(x) = \bb{f_i^{(2)}(x)}$ тоді і тільки тоді, коли вони утворюють на множині альтернатив однакові відношення переваги. 
\begin{teor}
Множина або вектор критеріїв $ \bb{f_i^{(1)}(x)}$ еквівалентний множині критеріїв $\bb{f_i^{(2)}(x)}$ тоді і тільки тоді, коли існує деяке монотонне перетворення $w$, яке переводить область значень кожного з критеріїв одної множини у область значень відповідного критерію з іншої множини.
\end{teor}
\begin{equation}
•
\end{equation}